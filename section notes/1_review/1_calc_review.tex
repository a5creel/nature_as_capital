\documentclass{article}
\usepackage[margin=1in]{geometry} 
\usepackage{amsmath,amsthm,amssymb,amsfonts, fancyhdr, color, comment, graphicx, environ}
\usepackage{xcolor}
\usepackage{mdframed}
\usepackage[shortlabels]{enumitem}
\usepackage{indentfirst}
\usepackage{hyperref}
\hypersetup{
    colorlinks=true,
    linkcolor=blue,
    filecolor=magenta,      
    urlcolor=blue,
}
\usepackage{pgfplots}
\pgfplotsset{width=10cm,compat=1.9}
\pgfplotsset{compat=1.17}
\usepackage{tikz}
\usepackage{caption}

\setlength{\parindent}{0pt}


%for headers 
\pagestyle{fancy}
\fancyhf{} % for header/footer

\lhead{Creel}
\rhead{ENV 795 - Nature as Capital}
\chead{\textbf{Calculus Review}}

\title{Calculus Review}
\author{Andie Creel for Nature as Capital}
\date{January 18th, 2023}



\begin{document}
\maketitle

% \section{Introduction}
% In these notes, I build up to solving an optimization problem using a Lagrangian. Optimization problems are common in economics because economists want to find how to maximize a benefit (ex. utility function) or minimize a harm (ex. damage function). The most common optimization problem in economics is maximizing a utility function subject to a constraint. 

\section{Common Derivative Rules}
Consider a function, $f(x)$, that depends on the variable $x$. Derivatives tell us the slope of the function at a specific point. The slope is the rate of change. \\

\textbf{Notation}: There are many ways to denote that you are taking an derivative. They are all (essentially) equivalent. I will switch between notations to keep the math neat.  
$$\frac{d}{dx}f(x) = f'(x) = f_x(x) = \Delta f(x)$$

\textbf{Constant rule:} Let c be a constant. 
$$\frac{d}{dx} cx = c$$

\textbf{Power rule:} $$\frac{d}{dx}x^n = n x^{n-1}$$

\textbf{Product rule:} $$\frac{d}{dx}(f(x) * g(x)) = f'(x)*g(x) + f(x)*g'(x)$$

\textbf{Quotient rule:} $$\frac{d}{dx} \frac{f(x)}{g(x)} = \frac{g(x)*f'(x)  - f(x) * g'(x)}{g(x)^2}$$
A rhyme that helps me remember the quotient rule: "Low d High minus High d Low all over the square of what's below".  \\

\textbf{Chain Rule:} $$\frac{d}{dx} f(g(x)) = f'(g(x)) * g'(x) $$

\textbf{Log Rules: } $$\frac{d}{dx}\log(x) = \frac{1}{x}$$

I am not sure if you'll need the following log rules for this course but wanted to include them here in case!

$$\log(x*y) = \log(x) + \log(y)$$ 

$$\log(\frac{x}{y}) = \log(x) - \log(y) $$

In this course, unless it's stated otherwise, you can assume all logs are "base e", i.e., 
$$\log_e(x) = \log(x) = \ln(x).$$ 


\textbf{Negative Exponent Rule:} Derivatives of functions with negative exponents follow the same power rule. \\
$$f(x) = \frac{1}{x^2} = x^{-2}$$
You could find the derivative of $f(x)$ using the quotient rule (because it is a fraction) or the power rule. \\

\textit{Quotient rule:} 
$$f'(x) = \frac{x^2 * 0 - 1*2x}{x^4} = \frac{-2x}{x^4} = \frac{-2}{x^3} = -2x^{-3}$$

\textit{Power rule: }
$$f'(x) = -2x^{-3}$$

With negative exponents, I find the power rule much easier. \\



\textbf{Partial Derivatives:} An function can depend on more than one variable, i.e., $U(x,y)$. You can take the derivative of that function with respect to a single variable and treat the other as a constant.\\

\textit{Example:} 
$$U(x,y) = x^{1/3} * y^{2/3}$$
$$U_x(x,y) = \frac{1}{3}x^{-2/3}*y^{2/3}$$
$$U_y(x,y)= \frac{2}{3}x^{1/3}y^{-1/3}$$



\subsection{Examples}

Example 1, power rule:\\ 
$x$ is a variable. \\
$$f(x) = 3 + x^2 + 4x^3$$
$$f'(x) = 2x + 12x^2$$\\

Example 2, logistic growth: \\
$r$ and $K$ are constants. $x$ is a variable. \\
$$g(x) = rx(1 - \frac{x}{K})$$
$$g(x) = rx - \frac{rx^2}{K} $$
$$g'(x) = r - \frac{2rx}{K}$$ \\

Example 3, profit function: \\
$a$, $b$ and $c$ are constants. $x$ and $w$ are variables. \\
$$\pi(x,w) = aw - \frac{b}{2}w^2 - c \frac{w}{x}$$
$$\frac{d \pi(x,w)}{dx} = c w x^{-2} $$
$$\frac{d \pi(x,w)}{d w} = a - bw - \frac{c}{x}$$


\section{Finding Max \& Min of Functions / Unconstrained Optimization}
We use derivatives to find the maximum or minimum of a function. Recall that derivatives tell us the slope of a function at a specific point. \\

At a maximum or minimum, the slope of the function will be zero. Therefore, to find the maximum or minimum of a function, we can:
\begin{enumerate}
    \item Find the derivative of the function, $f'(x)$.
    \item Set the derivative of the function equal to zero, $f'(x) = 0$.
    \item Solve for $x$. This value of $x$ maximizes the function $f(x)$.
    \item Plug the $x$ into $f(x)$ to find the maximum or minimum value of $f(x)$.
\end{enumerate}

\subsection{Example One}
Consider $g(x) = x - x^2$\\


\begin{tikzpicture}
\begin{axis}[axis lines = left,
    xlabel = \(x\),
    ylabel = {\(g(x)\)}]
\addplot[color=black,
    domain=0:1]
    {x-x^2};
\end{axis}
\end{tikzpicture}

Immediately we can see that the maximum of $g(x)$ is 0.25 and occurs at $x = 0.5$. However, let's solve for it following the steps above. 

\begin{enumerate}
    \item $g'(x) = 1 - 2x$
    \item $1 - 2x = 0$.
    \item $x = \frac{1}{2}$
    \item $g(\frac{1}{2}) = \frac{1}{2} - \frac{1}{2}^2= 1/2 - 1/4 = 1/4$.
\end{enumerate}

\subsection{Example two}
Consider the manager of a fishery. She knows that the fish population grows at the rate $$g(x) = 0.5x(1 - x/100),$$ where $x$ is the population of fish. She wants to find the population level that will lead to the maximum growth rate. How do you solve this problem?\\

First, rewrite $g(x)$
$$g(x) = \frac{1}{2}x - \frac{x^2}{200}$$

1) Find derivative: 
$$g'(x) = 1/2 - x/100$$

2) Set equal to zero:
$$1/2 -x/100 = 0$$

3) Solve for x: 
$$x = 50$$

4) Plug back into $g(x)$:
$$g(50) = 1/2*50 - 50^2/200 = 25 - 12.5 = 12.5$$\\

\begin{tikzpicture}
\begin{axis}[axis lines = left,
    xlabel = \(x\),
    ylabel = {\(g(x)\)}]
\addplot[color=black,
    domain=0:100]
    {.5*x - x^2/200};
\end{axis}
\end{tikzpicture}


% \section{Constrained Optimization}


    


\end{document}