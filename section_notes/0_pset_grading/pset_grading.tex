\documentclass{article}
\usepackage[margin=1in]{geometry} 
\usepackage{amsmath,amsthm,amssymb,amsfonts, fancyhdr, color, comment, graphicx, environ}
\usepackage{xcolor}
\usepackage{mdframed}
\usepackage[shortlabels]{enumitem}
\usepackage{indentfirst}
\usepackage{hyperref}
\hypersetup{
    colorlinks=true,
    linkcolor=blue,
    filecolor=magenta,      
    urlcolor=blue,
}
\usepackage{pgfplots}
\pgfplotsset{width=10cm,compat=1.9}
\pgfplotsset{compat=1.17}
\usepackage{tikz}
\usepackage{caption}

\setlength{\parindent}{0pt}


%for headers 
\pagestyle{fancy}
\fancyhf{} % for header/footer

\lhead{Creel}
\rhead{ENV 795 - Nature as Capital}
\chead{\textbf{Utility}}

\title{Grading Policy}
\author{Andie Creel for Nature as Capital}
% \date{February 13th, 2023}

\begin{document}
\maketitle

\section{General}
Grades should be thought of as a signal of how you're doing in the course. Here is how to interpret my signals. I will separate problem sets into subsections. Each subsection (which will typically be a questions) will get a grade out of 10. 
\begin{itemize}
    \item 10: Excellent work, went above and beyond. It will be very rare to get 10s, you should think of a 9 as 100\% and a 10 as extra credit.
    \item 9: Good work and correct answer!
    \item 8: Almost there, a few core aspects missing. If you want to do natural capital work as your job, talk to me about what you could improve in office hours. If you're just here to get the concepts, don't stress it.
    \item 7: Missing some core concepts (but don't be discouraged). Talk to me in office hours so we can solidify what you missed and you feel comfortable with the concepts moving forward! 
    \item 5: You made an attempt, but were missing the core concept. Definitely talk to me in office hours about this question! 
    \item 0: You didn't attempt or came close to not attempting. 
    
\end{itemize}

\section{Problem Set One}
Each question was graded out of 10. The problem set is out of 50 points. 

% \section{Problem Set Two}
% TBD

% \section{Problem Set Three}
% TBD

% \section{Problem Set Four}
% TBD

\end{document}