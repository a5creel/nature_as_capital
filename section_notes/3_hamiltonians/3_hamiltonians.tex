\documentclass{article}
\usepackage[margin=1in]{geometry} 
\usepackage{amsmath,amsthm,amssymb,amsfonts, fancyhdr, color, comment, graphicx, environ}
\usepackage{xcolor}
\usepackage{mdframed}
\usepackage[shortlabels]{enumitem}
\usepackage{indentfirst}
\usepackage{hyperref}
\hypersetup{
    colorlinks=true,
    linkcolor=blue,
    filecolor=magenta,      
    urlcolor=blue,
}
\usepackage{pgfplots}
\pgfplotsset{width=10cm,compat=1.9}
\pgfplotsset{compat=1.17}
\usepackage{tikz}
\usepackage{caption}

\setlength{\parindent}{0pt}


%for headers 
\pagestyle{fancy}
\fancyhf{} % for header/footer

\lhead{Creel}
\rhead{ENV 795 - Nature as Capital}
\chead{\textbf{Hamiltonians}}

\title{Section Three - Hamiltonians}
\author{Andie Creel for Nature as Capital}
\date{February 20th, 2023}

\begin{document}
\maketitle

\section{Introduction}
Recall from class that achieving sustainable development (non-declining welfare for future generations) we need to measure capital stocks, flows of welfare from those stocks, and the distribution of owning/accessing those stocks across different people. \\

\textbf{Current Value Hamiltonians} are a tool we use to measure the \textit{flow} of welfare from a \textit{capital stock} in the current time period. It is an annuity value. It's equation is,

$$\delta V = H = W(s, x(s)) + \lambda \dot s$$
$$ H = divdends + capital\ gains$$

 where $H$ is the value of flows from a stock in a time period, $W(s, x(s))$ is the welfare (or profit) from that time period given the stock, $s$ and management decisions (\textbf{choice variable}), $x(s)$ in that time period, $\lambda$ is the shadow price (capital price) and $\dot s$ is the change in the capital stock in that time period. If the stock was a population, we could model $\dot s$ as logistic growth, the Allee affect, a Holling disk equation, or any other growth equations we've seen to model how a population changes. 

\subsection{Welfare/Well-being/Net Benefits/Profits}

\textbf{In any time period}, we can measure the welfare gained from a stock. Sometimes, the welfare from a stock in a given time period will only be the profit from selling the harvest of a stock. Other times, it will be a more complicated equation that accounts for existence value. \\

For simplicity, let's consider an example where the only welfare we're interested in is the profit gained from harvesting a stock in a time period. In class, we say this written as  
$$W(s, x(s)) = p*h(x(s)) - c * h(x(s)) $$
where $h(x(s))$ was the harvest decision given the value of our choice variable, $x(s)$. \\

Let's simplify this equation as say that our choice variable, $x(s)$, is our harvest decision. Meaning, if $x(s) = 4$, we've harvested 4 fish in that time period,  

$$W(s, x(s)) = p*x(s) - c * x(s) $$
where $p$ is the market price, $x(s)$ is our harvest decision and $c$ is the cost of harvesting a fish (we've assumed costs are linear). \\

Therefore, this welfare equation measures the pay out we get in a given time period (i.e., dividends). 

$$dividends = W(s, x(s))$$

\subsection{Growth of Stock}
Recall that dot notation, $\dot s$, means the way a stock grows through time. It's the growth rate. We went over many ways a stock can grow in the ecological modeling notes. \\

\subsection{Capital/Shadow Price}
The shadow price, $\lambda$, the capital price. $\lambda$ measures the value of having one more unit of stock (ex. the value of having one more fish). \\

Therefore we can think of our capital gains (or loss) in any time period as the shadow/capital price multiplied by the capital growth in that time period.

$$capital\ gains = \lambda \dot s $$

The shadow/capital price is why the current value Hamiltonian is so important to natural capital because it is a non-market price for a non-market capital asset. However, it's found using traditional accounting measures. 

\subsection{Finding $\lambda$}
Sometimes you will be able to directly solve for $\lambda$. However, it's not always possible. What is always possible is some algebraic manipulation of our original equation 

$$\delta V = H = W(s, x(s)) + \lambda \dot s$$
$$V = \frac{W(s, x(s)) + \lambda(s) \dot s  }{\delta}.$$
In this equation, we may not know $V$ or $\lambda$. One equation, one unknown, which is a problem because we want to solve for $\lambda$ to get the capital asset price! \\

Luckily, in the video this week, we learned that the marginal value is defined as the price, 

$$\frac{dV}{ds} \equiv p.$$ 

Taking the derivative of the equation above is how Eli gets the price equation as 
$$ \lambda(s) = \frac{W_s + \frac{\partial \lambda}{\partial s}\dot s }{\delta - \dot s_s}.$$

Which is one equation and one(ish) unknown. The CAPn solves the above equation numerically. 


\end{document}