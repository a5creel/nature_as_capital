\documentclass[12pt]{article}

\usepackage[paper=letterpaper,margin=2.5cm]{geometry} % Set Margins

%% Math and math fonts
\usepackage{amsmath, amsthm, amssymb, amsfonts}
\usepackage{bbm} % for \mathbbm{1}

% date
\usepackage[nodayofweek]{datetime}

% Color
\usepackage{color, xcolor}

% Misc
\usepackage{environ}  % \collect@body in asmmath
\usepackage{graphicx} % \includegraphics options
\usepackage{mdframed} % text boxes
\usepackage{indentfirst} % Indent first paragraph after section header
\usepackage[shortlabels]{enumitem} % Control enumerate items with [(a)]
\usepackage{comment} % Comments
\usepackage{fancyhdr} % Headers and footers

% Tables
\usepackage{array}

% Sub-figures and figure placement
\usepackage{caption}
\usepackage{subcaption}
\usepackage{float} 

% Graphing
\usepackage{pgfplots}
\pgfplotsset{compat=1.17}
\usepackage{tikz}

% Title Placement
\usepackage{titling}
\setlength{\droptitle}{-6em}

%set indent to 
\setlength{\parindent}{0pt}

%for headers 
\pagestyle{fancy}

\lhead{Creel}
\chead{Midterm answers}
\rhead{Nature as Capital}

\title{Reviewing What People Missed}
\author{Andie Creel}

% Hyper refs
\usepackage{hyperref}
\hypersetup{
    colorlinks=true,
    linkcolor=blue,
    urlcolor  = blue,
    filecolor=magenta,      
    urlcolor=blue,
    citecolor = blue,
    anchorcolor = blue
}

% % Citation management
\usepackage{natbib}
\bibliographystyle{abbrvnat}
\setcitestyle{authordate,open={(},close={)}}

\begin{document}
\maketitle

\textbf{Shadow price:} The shadow price value equals the value of an additional unit of the constraint (\textit{ie} budget or stock). 
\begin{align}
    \frac{\partial \mathcal{L}}{\partial B} = \lambda
\end{align}


\textbf{Number of Choice Variables:} If you are going to consume the entirety of a resource in three time periods, that means you only have 2 choice variables. You can choose how much to consume in time period one, time period two, and then those \textit{two} choices will determine what you consume in time period three. This is important to be clear about in policy contexts, because if a policymaker thinks they have a third choice that proves that they are not thinking clearly about how the resource is being exhausted. \\

\textbf{Tradeable Permits and Tax:} If a quota is set optimally and a tax is set optimally, then you will arrive at the same equilibrium set of prices and quantity. 

\section{Conrad 3.4}
\begin{align}
    \mathcal{L} = \sum_0^\infty [\rho^t ln(Y_t) + \rho^{t+1} \lambda_{t+1} [X_t + r X_t(1-X_t/K)^\beta - Y_t - X_{t+1}]]
\end{align}

FOCs:
\begin{align}
    \frac{\partial \mathcal{L}}{\partial Y} = \frac{1}{Y_t}- \rho \lambda_{t+1} = 0\\
    \frac{\partial \mathcal{L}}{\partial X} = \rho^{t+1}\lambda_{t+1}[1+ G'(X_t)] + \rho^t \lambda_t(-1) = 0 \label{foc_2}\\
    \frac{\partial \mathcal{L}}{\partial \rho^{t+1}\lambda_{t+1}} = X_t + r X_t(1-X_t/K)^\beta - Y_t - X_{t+1} = 0 
\end{align}

When $r = 1, K = 1, \beta = 1/2$, then  
\begin{align}
    G(X) = X(1-X)^{1/2}\\
    G'(X) = (1-X)^{1/2} + X [-\frac{1}{2}(1 - X)^{-1/2}] \label{g_prime}
\end{align}

So you can plug \ref{g_prime} into \ref{foc_2} and you have your three FOCs. 

\end{document}


\end{document}