\documentclass[12pt]{article}

\usepackage[paper=letterpaper,margin=2.5cm]{geometry} % Set Margins

%% Math and math fonts
\usepackage{amsmath, amsthm, amssymb, amsfonts}
\usepackage{bbm} % for \mathbbm{1}

% date
\usepackage[nodayofweek]{datetime}

% Color
\usepackage{color, xcolor}

% Misc
\usepackage{environ}  % \collect@body in asmmath
\usepackage{graphicx} % \includegraphics options
\usepackage{mdframed} % text boxes
\usepackage{indentfirst} % Indent first paragraph after section header
\usepackage[shortlabels]{enumitem} % Control enumerate items with [(a)]
\usepackage{comment} % Comments
\usepackage{fancyhdr} % Headers and footers

% Tables
\usepackage{array}

% Sub-figures and figure placement
\usepackage{caption}
\usepackage{subcaption}
\usepackage{float} 

% Graphing
\usepackage{pgfplots}
\pgfplotsset{compat=1.17}
\usepackage{tikz}

% Title Placement
\usepackage{titling}
\setlength{\droptitle}{-6em}

%set indent to 
\setlength{\parindent}{0pt}

%for headers 
\pagestyle{fancy}

\lhead{Creel}
\chead{SCC}
\rhead{Nature as Capital}

\title{Week Thirteen -- Social Cost of Carbon}
\author{Andie Creel}

% Hyper refs
\usepackage{hyperref}
\hypersetup{
    colorlinks=true,
    linkcolor=blue,
    urlcolor  = blue,
    filecolor=magenta,      
    urlcolor=blue,
    citecolor = blue,
    anchorcolor = blue
}

% % Citation management
\usepackage{natbib}
\bibliographystyle{abbrvnat}
\setcitestyle{authordate,open={(},close={)}}

\begin{document}
\maketitle

\section{Social Cost of Carbon}
The DICE model started as a modeling exercise but really took off as \textit{the} model. Keep in mind that it relies on Negishi weights which are the inverse of marginal utility of money. Negishi weights up-weight the rich and down-weights the poor because wealthy people tend to have a lower marginal utility for money. 

\begin{align}
    SWF = \max_{c(t)} \sum_t U(C(t), P(t)) (1 + \rho)^{-1}\\
    U(C,P) = \frac{P(C^{1- \alpha} - 1}{1 - \alpha}\\
    r = \rho + \alpha g \label{ramsey}
\end{align}
Where $C$ is consumption and $P$ is population and $\rho$ is the pure rate of social time preference, $r$ is the discount rate. Not that \ref{ramsey} is the Ramsey rule, which we have seen before. 

Production is written 
\begin{align}
    Q(t) = \Omega(t)A(t)K(t)^\gamma P(t)^{1-\gamma}
\end{align}
$\gamma$ is the percent change in capital wrt change in capital. K is produced capital. The population is the labor supply. This is Cobb-Douglas form which means we've assumed constant returns to scale. $A$ is the Solow residual \textit{aka} total factor productivity. $\Omega$ is where the climate effects enter. 
\begin{align}
    C(t) = Q(t) - I(t)
\end{align}
The consumer is just production $Q$ minus investment.
\begin{align}
    K(t) = (1 - \delta_K) K(t-1) + I(t)
\end{align}
where $\delta_K$ is the depreciation of capital. \\

Emissions are written 
\begin{align}
    E(t) = [1 - \mu(t)] \sigma(t) Q(t)
\end{align}
$\mu$ is the investment in emissions reduction. $\sigma$ assumes no emissions control but assumes market powers will make it so that production is less emissions-intensive. 
\begin{align}
    M(t) = \beta E(t) + (1 - \delta_M) M(t-1)
\end{align}
where $M$ the stock of emissions in the atmosphere. Notice is=t works just like the capital stock. 

\begin{align}
    F(t) = f(M(t))
\end{align}
where $F(t)$ is the effect of emissions on climate. \\

Nordhaus then creates a damage function 
\begin{align}
    d(t) = 0.0133 (\frac{T(t)}{3})^2 Q(t)
\end{align}
damages don't change investment or the depreciation of capital. It does affect productivity 
\begin{align}
    \Omega(t) = \frac{(1- b_1) \mu(t)^{b_3}}{1 + d(t)}
\end{align}
Eli: we're over-investing in getting a damage function rather than figuring out how to have damages affect investment and depreciation. 

\begin{align}
    \max_{C(t), \mu(t)} = \sum_t \frac{P(t) (C(t)^{1- \alpha} - 1)}{1 - \alpha} (\frac{1}{1 + \rho})^t
\end{align}

Finally, we can plug loads of stuff in. 
\begin{multline}
    \Delta K = (1 -\delta_K) K(t-1) + \frac{(1- b_1) \mu(t)^{b_3}}{1 + d(t)} A(t-1) K(t-1)^\gamma P(t-1)^\gamma - C(t-1)P(t -1 ) - K(t-1)
\end{multline}

\begin{multline}
    \Delta M = (1 - \delta_M) M(t-1) + \beta (1 - \mu(t -1)) \sigma(t-1) \frac{(1 - b_1) \mu(t-1)^{b_2}}{1 + \delta(M(t-1))}A(t-1) K(t-1)^\gamma P(t-1)^(1 - \gamma) - M(t)
\end{multline}

\begin{align}
    H = \frac{P(t) (C(t)^{1 - \alpha} - 1)}{1 - \alpha} \frac{1}{1+ \rho} + \lambda(t) \frac{1}{1+ \rho} \Delta K + \eta(t) \frac{1}{1+ \rho} \Delta M  \label{hamil}
\end{align} 
\ref{hamil} is just a Hamiltonian with two stocks, $K$ and $M$. He solved it using the expensive version of solver. Eli described this first DICE model as a term project on steroids. 

\section{EPA's "textbook" on SCC}
Best methods available for calculating SCC: \url{https://www.epa.gov/system/files/documents/2022-11/epa_scghg_report_draft_0.pdf}

As of April of 2023, it was still being peer-reviewed.


\end{document}